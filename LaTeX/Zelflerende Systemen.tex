
\documentclass[a4paper]{article}

\title{Zelflerende Systemen}
\author{Steven Bronsveld en Thijs van Loenhout}

\begin{document}
\maketitle

\renewcommand{\contentsname}{Inhoud}

\tableofcontents

\section{Inleiding}

\section{Wat zijn zelflerende computersystemen?}



\subsection{Inleiding}
Awef (hoe persoonlijk willen we dit maken? We moeten er rekening mee houden dat dit verslag als het goed is de verwezenlijking van geweldigheid gaat worden. Misschien wordt het bij de universiteit bekeken en zou het raar zijn als we beginnen met “hallo, wij zijn Steven en Thijs van het Candea College in Duiven!”)


\subsection{Algoritmes}

\subsection{Zelflerend?}

\subsection{Machine Learning}
Een zelflerend systeem is een algoritme gebaseerd op machine learning. Machine learning wordt door Arthur Samuel, een pionier op dit gebied, gedefinieerd als: "A field of study that gives computers the ability to learn without being explicitly programmed.”\cite{ArthurSamuel}. In tegenstelling tot de eerder genoemde algoritmes is een zelflerend systeem instaat zichzelf te verbeteren. Dit betekent ook gelijk dat het algoritme niet perfect begint en dat ook nooit zal worden (zie deelvraag 3). Daarom is het nodig om het systeem te trainen. 

\subsection{Trainen}
Om een zelflerend systeem instaat te stellen zichzelf te verbeteren moet het algoritme “weten” wat de gewenste output is. Een algoritme kan niet zomaar het gewenste resultaat geven. Om met tot de gewenste output te komen is er dus een periode van training nodig. Hierbij krijgt het systeem input waarvan de gewenste output bekend is. Op basis hiervan kan het zichzelf vervolgens verbeteren, de manier waarop dit gebeurd bespreken we in het volgende kopje. Er zijn drie manieren van trainen te onderscheiden: supervized learning, unsupervised learning en reinforcement learning. 

\subsubsection{Supervised Learning}

\subsubsection{Unsupervised Learning}

\subsubsection{Reinforcement Learning}





\subsection{Evolutionary Systems}

\subsection{Conclusie}


\bibliographystyle{unsrt}
\bibliography{sample}

\end{document}
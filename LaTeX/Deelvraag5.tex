\section{Welke toepassingen hebben systemen die gebruik maken van een zelflerend algoritme?}
Kunstmatige intelligentie klinkt misschien iets uitsluitend uit sciencefiction, maar in werkelijkheid kent het al vele toepassingen in de hedendaagse wereld. Zover als zelfdenkende robots gaat het echter nog niet. Dit is een voorbeeld van strong AI (strong Artificial Intelligence).

\subsection{Weak AI}
Er kan een onderscheid gemaakt worden tussen weak AI en strong AI. Dit zegt niet zozeer iets over de denkkracht van het systeem, maar eerder over de manier waarop het met informatie omgaat.
De meeste voorbeelden van hedendaagse AI vallen in de eerste categorie. Weak AI is ontworpen voor een specifieke taak. Persoonlijke assistenten als Siri en Cortana zijn hier goede voorbeelden van. Ze zijn ontworpen om te functioneren binnen een van tevoren bepaald gebied. Zodra je iets tegen Siri zegt wat niet vergelijkbaar is met de dingen binnen dit gebied, dan zal zij niet in staat zijn goed te reageren. Er is geen sprake van echte intelligentie of bewustzijn.
Persoonlijke assistenten werken met voice recognition. Wanneer de gebruiker iets zegt, vergelijken de assistenten dit met dingen die ze kennen. Ze kiezen uit wat het meest vergelijkbaar is en geven op basis van deze vergelijking een reactie. Deze vergelijking maken is kenmerkend voor weak AI. Deze soort AI werkt dus met supervised learning.

\subsection{Strong AI}
De AI die dichter in de buurt komt van de robots uit sciencefiction is strong AI. Het streven hierbij is een programma te maken vergelijkbaar met een mensenbrein. \cite{Searle} Een AI als deze zou nieuwe informatie moeten kunnen interpreteren. Hiervoor moet het, naast vergelijken, kunnen associëren. Een simpel voorbeeld is zeggen dat je de volgende dag om acht uur op wil staan. Een weak AI zal waarschijnlijk niks met deze informatie doen, terwijl een strong AI het initiatief zou kunnen nemen om de wekken op acht uur te zetten. Strong AI maakt dus gebruik van unsupervised learning.
Deze vorm van artificial intelligence vereist echter nog veel onderzoek.

\subsection{Artificial General Intelligence (AGI)}
Natuurlijk kan een programma ook voor meerdere taken toepasbaar zijn zonder dat het bewustzijn heeft, zoals strong AI graag zou zien. In dit geval wordt gesproken van artificial general intelligence. Dit is AI die zichzelf kan leren verschillende dingen te doen. Een voorbeeld is Deep Q, een deep artificial neural network. Deep Q leerde zichzelf een bepaald Atari 2600 spel te spelen. Toen dit lukte en de onderzoekers 48 andere spellen aan het ANN gaven, was Deep Q in staat ook deze spellen, waarvoor hij niet geprogrammeerd was, te kunnen spelen. \cite{DeepQ}
AGI is al een stuk verder dan Strong AI is, zeker met de recente ontwikkeling van Google DeepMind. \cite{DeepMind1}

\subsection{Verdere toepassingen}
Machine Learning kan hulp bieden bij vele taken op heel veel gebieden. Dit komt doordat sommige vraagstukken te groot zijn voor een mensenbrein of te veel berekeningen vereisen. Denk bijvoorbeeld aan het maken van schoolroosters. De optimale roosters vinden voor honderden leerlingen en docenten is een opdracht die voor een mens, zonder hulp van een computer, haast niet te doen is. De computer echter kan veel sneller de opties langsgaan om te zoeken naar het optimum. De mens moet dan slechts nog aangeven waardoor dit optimum wordt bepaald.
Ook in de financiële sector zijn vele toepassingen te noemen. Neem bijvoorbeeld de aandelenmarkt. Continu vinden stijgingen en dalingen plaats van bepaalde waardes en na een tijdje kan het teveel worden voor een mens. Een computer is echter in staat veel meer waardes te interpreteren en te vergelijken. Daarom worden er programma’s getraind om de loop van deze markt te voorspellen.
Het aantal voorbeelden dat hier gegeven van worden is ontzettend groot. In vrijwel elk gebied is wel iets te bedenken waarin een computer, een AI hulp kan bieden.

\subsection{Conclusie}
Zelflerende systemen worden ingezet voor taken die voor de mens te groot, te moeilijk of te intensief worden. Er kan onderscheid gemaakt worden tussen zulke systemen op basis van toepasbaarheid (voor een enkele taak of voor een onbepaald aantal taken) en de manier waarop het met informatie omgaat.